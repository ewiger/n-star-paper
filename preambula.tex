%\documentclass[sigconf,natbib=false]{acmart}
% ---------------------------
% \usepackage[utf8]{inputenc}
\usepackage[english]{babel}

% \babelprovide[import]{hebrew}
%\babelprovide[onchar=fonts ids]{hebrew}
% \babelfont[hebrew]{rm}[SlantedFont = SILEOT]{SILEOT.ttf}

\usepackage{csquotes}
% ---

\usepackage{datetime}
\usepackage{hyperref}
\usepackage{float}
\usepackage{graphicx}
\usepackage{subcaption}
\usepackage[hyperref = true,
  backref = true,
  backend = bibtex,
  style = nature,
	sorting = none,
]{biblatex}

% FIXME:
\usepackage{mathalpha}
\usepackage{mathtools}
% \usepackage{stix2}

\usepackage{amsmath}
\usepackage{amsthm}
\usepackage{amssymb}
\usepackage{stmaryrd}

\usepackage{tikz-cd} % For commutative diagrams

% https://mirror.foobar.to/CTAN/macros/latex/contrib/mathalpha/doc/mathalpha-doc.pdf
% \DeclareMathAlphabet\mathbfcal{LS2}{stixcal}{b}{n}
\DeclareMathAlphabet\mathbfcal{OMS}{cmsy}{b}{n}

% ---------------------------
\usepackage[shortlabels]{enumitem} 
\usepackage{listings}
\usepackage{color}
\usepackage{appendix}
\usepackage[normalem]{ulem}
\usepackage{makecell}

%-----------------------------
% sub-"enumerate" with enumitem
\newlist{legal}{enumerate}{10}
\setlist[legal]{label*=\arabic*.}
%-----------------------------

\date{\displaydate{date}}
\date{\today\\v1.0.0}

\definecolor{somegreen}{rgb}{0,0.6,0}
\definecolor{somegray}{rgb}{0.5,0.5,0.5}
\definecolor{mymauve}{rgb}{0.58,0,0.82}

\lstset{ 
  backgroundcolor=\color{white},   % choose the background color; you must add \usepackage{color} or \usepackage{xcolor}; should come as last argument
  basicstyle=\footnotesize,        % the size of the fonts that are used for the code
  breakatwhitespace=false,         % sets if automatic breaks should only happen at whitespace
  breaklines=true,                 % sets automatic line breaking
  captionpos=b,                    % sets the caption-position to bottom
  commentstyle=\color{somegreen},    % comment style
  deletekeywords={...},            % if you want to delete keywords from the given language
  escapeinside={\%*}{*)},          % if you want to add LaTeX within your code
  extendedchars=true,              % lets you use non-ASCII characters; for 8-bits encodings only, does not work with UTF-8
  firstnumber=1000,                % start line enumeration with line 1000
  frame=single,	                   % adds a frame around the code
  keepspaces=true,                 % keeps spaces in text, useful for keeping indentation of code (possibly needs columns=flexible)
  keywordstyle=\color{blue},       % keyword style
  language=C++,		               % the language of the code
  morekeywords={*,...,contract,function,returns},            % if you want to add more keywords to the set
  numbers=left,                    % where to put the line-numbers; possible values are (none, left, right)
  numbersep=5pt,                   % how far the line-numbers are from the code
  numberstyle=\tiny\color{somegray}, % the style that is used for the line-numbers
  rulecolor=\color{black},         % if not set, the frame-color may be changed on line-breaks within not-black text (e.g. comments (green here))
  showspaces=false,                % show spaces everywhere adding particular underscores; it overrides 'showstringspaces'
  showstringspaces=false,          % underline spaces within strings only
  showtabs=false,                  % show tabs within strings adding particular underscores
  stepnumber=2,                    % the step between two line-numbers. If it's 1, each line will be numbered
  stringstyle=\color{mymauve},     % string literal style
  tabsize=2,	                   % sets default tabsize to 2 spaces
  title=\lstname                   % show the filename of files included with \lstinputlisting; also try caption instead of title
}

\newtheoremstyle{mydef}
{\topsep}{\topsep}%
{}{}%i
{\bfseries}{}
{\newline}
{%
  \rule{\textwidth}{0.4pt}\\*%
  \thmname{#1}~\thmnumber{#2}\thmnote{\ -\ #3}.\\*[-1.5ex]%
  \rule{\textwidth}{0.4pt}}%

\theoremstyle{mydef}
\newtheorem{definition}{Definition}
\newtheorem{theorem}{Theorem}
\newtheorem{claim}[theorem]{Claim}
\newtheorem{proposition}[theorem]{Proposition}
\newtheorem{lemma}[theorem]{Lemma}
\newtheorem{corollary}[theorem]{Corollary}
\newtheorem{conjecture}[theorem]{Conjecture}
\newtheorem{observation}{Observation}
\newtheorem{principle}{Principle}
\newtheorem*{example}{Example}
\newtheorem*{remark}{Remark}
\newtheorem*{axiom}{Axiom}

% also see options for fonts in latex 
% https://tex.stackexchange.com/questions/58098/what-are-all-the-font-styles-i-can-use-in-math-mode
\newcommand*{\numberset}[1]{\mathbb{#1}} 
\newcommand{\nat}{\numberset{N}}
\newcommand{\integers}{\numberset{Z}}
\newcommand{\rationals}{\numberset{Q}}
\newcommand{\reals}{\numberset{R}}
\newcommand{\cset}{\mathcal{C}}
\newcommand{\bin}{\mathcal{B}}
\newcommand{\cbin}{\numberset{B}}
\newcommand{\pset}{\mathcal{P}}
\newcommand{\yset}{\mathcal{Y}}
\newcommand{\rset}{\mathcal{R}}
\newcommand{\porder}{\mathbb{P}}
\newcommand{\cont}{\mathfrak{c}}
\newcommand{\mcar}{\mathfrak{m}}
\newcommand{\cesef}{\mathrm{E}_{SEF}} % countable list of all labels of eq classes over all fin formulas
\newcommand{\uesef}{\mathcal{E}_{SEF}} % uncountable list of all labels of eq classes over all infinite formulas
\newcommand{\ufsef}{\mathcal{F}_{SEF}} % uncountable list of fair formulas
\newcommand{\uusef}{\mathcal{U}_{SEF}} % uncountable list of unfair formulas
\newcommand{\vsef}{\mathbb{S}} % syntactically valid SEFs
\newcommand{\vhsef}{\mathbb{T}} % cumulative hierarchy syntactically valid SEFs

\newcommand{\lsef}{L_{SEF}}  % countable list of all finite formulas
\newcommand{\ulsef}{\mathcal{L}_{SEF}}  % uncountable list of all infinite (countable) formulas

\newcommand{\psf}{\mathfrak{F}} % probability space: field of sets - events

% model theory
\newcommand{\lsign}{\mathbfcal{L}}  % L-signature or L language
\newcommand{\Lf}{\mathbfcal{F}}  % Functions
\newcommand{\Lr}{\mathbfcal{R}}  % Relations
\newcommand{\Lc}{\mathbfcal{C}}  % Constants
\newcommand{\LsT}{\mathbfcal{T}}  % L-theory
\newcommand{\LsA}{\mathbfcal{A}}  % L-structure
\newcommand{\LsB}{\mathbfcal{B}}  % L-structure
\newcommand{\LsN}{\mathbfcal{N}}  % L-structure
\newcommand{\LsM}{\mathbfcal{M}}  % L-structure
\newcommand{\LsQ}{\mathbfcal{Q}}  % L-structure
\newcommand{\LsV}{\mathbfcal{V}}  % L-structure
\newcommand{\LsS}{\mathbfcal{S}}  % L-structure

% for boolean lagebra signature
\newcommand{\bF}{\textbf{0}}  % algebra structure
\newcommand{\bT}{\textbf{1}}  % algebra structure

\newcommand{\notimplies}{\;\not\!\!\!\implies}
\newcommand{\substr}{\ \varepsilon\ }

% some extended hebrew notation
% \newcommand{\HebrewLetterAsMathSymbol}[1]{\text{\foreignlanguage{hebrew}{#1}}}
% \newcommand{\hebhe}{\HebrewLetterAsMathSymbol{ה}} % 5
% \newcommand{\hebvav}{\HebrewLetterAsMathSymbol{ו}} % 6
% \newcommand{\hebzayin}{\HebrewLetterAsMathSymbol{ז}} % 7  
% \newcommand{\hebchet}{\HebrewLetterAsMathSymbol{ח}} % 8
% \newcommand{\hebtet}{\HebrewLetterAsMathSymbol{ט}} % 9
% \newcommand{\hebtav}{\HebrewLetterAsMathSymbol{ת}} % 22


% function restriction
\newcommand\restr[2]{{% we make the whole thing an ordinary symbol
  \left.\kern-\nulldelimiterspace % automatically resize the bar with \right
  #1 % the function
  \vphantom{\big|} % pretend it's a little taller at normal size
  \right|_{#2} % this is the delimiter
  }}

% use quotes in math mode to define strings
%\DeclareMathSymbol{\mlq}{\mathord}{operators}{"}
%\DeclareMathSymbol{\mrq}{\mathord}{operators}{"}
\newcommand{\bos}{\text{``}}
\newcommand{\eos}{\text{''}}

% complement
\newcommand{\stcomp}[1]{{#1}^\complement}

% bij 
\newcommand\bij{\lhook\joinrel\twoheadrightarrow}

% cupdot disjoint union UPDATE: defined by stix2
\newcommand{\cupdot}{\mathbin{\mathaccent\cdot\cup}}
\newcommand{\fdot}{\mathbin{\mathaccent\cdot F}} % uncomputable fair

% Inserts \clearpage before \begin{appendices}
\BeforeBeginEnvironment{appendices}{\clearpage}
% Inserts \clearpage after \end{appendices}
%\AfterEndEnvironment{appendices}{\clearpage}

% Inserts \clearpage before every \section within appendices environment
%\AtBeginEnvironment{appendices}{\pretocmd{\section}{\clearpage}{}{}}{}

% doted curly wedge and vee
\newcommand{\cdwedge}{\dot{\curlywedge}}
\newcommand{\cdvee}{\dot{\curlyvee}}

% nand
\newcommand{\nand}{\barwedge}
