\pagebreak
\hspace*{\fill}\textit{For Mary and Emilia}

\pagebreak
\section{Preface}

Embarking on the exploration of a simple algorithm years ago marked the beginning of an extraordinary journey for me. This quest soon transformed not only into a hobby, but rather what I affectionately refer to as my favorite long-term relationship. Back in 2007, as a software engineer with practical experience in the industry, I found myself increasingly drawn to the theoretical aspects of mathematics, eventually developing a keen interest in Set Theory.

The concept of applying the CTA algorithm in an infinite context did not emerge suddenly. Instead, it was a gradual realization, spurred by my encounter with Cantor's diagonal argument (which to my shame I completely misunderstood at first and was able to fully comprehend only years after graduating in technical field). This personal discovery was a significant turning point, opening up the new and unknown realm of infinite for me. 

My initial intention was purely educational, aimed at expanding my understanding as much as possible. Over the years, this scholarly pursuit evolved. I gathered my scattered notes, gradually mustering the courage to compile them into a coherent manuscript. The objective was to produce a concise paper, offering a computer science perspective on the notion of infinities, particularly in relation to the continuum hypothesis.

However, as time passed, it became painfully clear through numerous trials and errors that a computer science lens alone was inadequate. It lacked the necessary precision not just for answering questions, but even for formulating them correctly. This realization necessitated a deeper, more fundamental approach to learning.

A pivotal moment in my journey was the acknowledgement that disproving the continuum hypothesis (CH) or proposing a naive theoretical example of its possibility was not a novel achievement. Paul Cohen's groundbreaking 1963 paper, which elegantly and rigorously demonstrated the independence of CH using the sophisticated method of forcing, was a humbling reminder of the complexity of these problems.

Despite these challenges, I remained motivated by the belief that my unique approach might still offer a fresh perspective or become a useful tool in the field. This paper is the result of years of learning, exploring, and refining these ideas.

Now, as I conclude this draft, I am ready to present my findings. This paper represents not just an academic endeavor, but also a personal journey from a software engineering background into the depths of mathematical theory.
